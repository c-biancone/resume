%%%%%%%%%%%%%%%%%%%%%%%%%%%%%%%%%%%%%%%%%%%%%%%%%%%%%%%%%%%%%%%%%%%%
% 
% Written by: Chris Biancone
%
% Important note:
% This document depends on the resume.cls class file for structuring
%
% Rev. 01/23/2023
%
%%%%%%%%%%%%%%%%%%%%%%%%%%%%%%%%%%%%%%%%%%%%%%%%%%%%%%%%%%%%%%%%%%%%

%-------------------------------------------------------------------
%	PACKAGES AND OTHER DOCUMENT CONFIGURATIONS
%-------------------------------------------------------------------

\documentclass{cv}  % Use the custom resume.cls style

\usepackage[fixed]{fontawesome5}    % for icons (monospaced)
\usepackage[hidelinks]{hyperref}
\usepackage{paracol}    % multicolumn layout
\usepackage{fancyhdr}   % footer
\usepackage[left=0.5in,top=0.5in,right=0.5in,bottom=0.5in,columnsep=0.5in]{geometry}

% define tab spacing
\usepackage{setspace}
\newcommand{\tab}[1]{\hspace{.2667\textwidth}\rlap{#1}}
\newcommand{\itab}[1]{\hspace{0em}\rlap{#1}}

% custom link underlining
\usepackage{contour}
\usepackage{ulem}
\renewcommand{\ULdepth}{1.8pt}
\contourlength{0.8pt}
\newcommand{\myuline}[1]{\uline{\phantom{#1}}\llap{\contour{white}{#1}}}

\usepackage{siunitx}
\sisetup{detect-all=true}   % ensure proper font weight
\sisetup{per-mode = repeated-symbol}    % ensure "/" unit delineator
\sisetup{range-phrase = --,range-units = brackets} % hyphenated ranges, SI syntax
\DeclareMathSymbol{\varOmega}{\mathalpha}{operators}{"0A}   % upright omega for ohms
\providecommand*{\Ohm}{\varOmega}   % for siunitx
\DeclareSIUnit\sq{\ensuremath{\Box}}    % sheet resistance
\DeclareSIUnit{\Siemens}{S}
\DeclareSIUnit{\torr}{Torr} % add torr to siunitix 3

% setup page
\begin{document}
\pagestyle{fancy}
\renewcommand{\headrulewidth}{0pt}
\renewcommand{\footrulewidth}{2pt}
\fancyfoot[C]{github.com/c-biancone/resume}

% \SetPageColorSpace{PANTONE}
% \definecolor{deepblue}{spotcolor}{PANTONE19-3847TCX, 1.0}

%-------------------------------------------------------------------
%	HEADER
%-------------------------------------------------------------------
\parbox[b]{0.25 \textwidth}{\linespread{0.75}\bold\HUGE Chris Biancone}
\hfill
\hspace{5pt}
\parbox[b]{0.2 \textwidth}{\Large{\bold{BS/MS Electrical \\ Engineering}}}
\hfill
\parbox[b]{0.26 \textwidth}{\RaggedRight\small\faPhone\thinspace (973) 896-0255 \href{mailto:chris.biancone@gmail.com}{\faEnvelope\thinspace\myuline{chris.biancone@gmail.com}} \setstretch{1.25}
\href{https://www.linkedin.com/in/c-biancone/}{\faLinkedin\thinspace\myuline{c-biancone}} \href{https://github.com/c-biancone}{\faGithub\thinspace\myuline{c-biancone}}}

% \hspace{3pt}
\vspace{-8pt}

{\color{lightgray}\rule{\linewidth}{2pt}\par}

\vspace{-3pt}

% resume body column structure
\columnratio{0.33}
\begin{paracol}{2}

%-------------------------------------------------------------------
%	ABOUT
%-------------------------------------------------------------------
\cvsection{About}

\RaggedRight
Fifth-year EE student at RIT proficient in analog and mixed signal electronics design and instruction. Experienced CAD tools such as C\=adence, Altium, and MATLAB for microelectronics, control system, and SDR system design. Currently conducting research into biophysical effects of DC magnetic fields and semiconducting properties of neurons for publication in a graduate paper.

I am an Isshin-Ryu black belt and enjoy hiking and mountain biking. I have recently combined a love of music with my knowledge in analog electronics design to design a low-tolerance differential RIAA preamplifier.

%-------------------------------------------------------------------
%	EDUCATION
%-------------------------------------------------------------------
\cvsection{Education}
{\large\bold{Rochester Institute of \\ Technology}}

{GPA: 3.7} \hfill 2019 -- Present \\
BS/MS Electrical Engineering with \\
MS focus in MEMS\\

% {\Large\bold{Pope John XXIII HS}}

% {GPA: 4.26} \hfill 2015 - 2019 \\
% MIT Zero Robotics \\
% Captain FTC Robotics

%-------------------------------------------------------------------
%	SKILLS
%-------------------------------------------------------------------
\cvsection{Skills}
\begin{description}
    \item[\faLaptop] \hspace{0.25em} C\=adence Virtuoso, Altium, COMSOL, SPICE, Solidworks
\end{description}
\begin{description}
    \item[\faCode] \hspace{0.25em} C++, Python, MATLAB, \LaTeX{}
\end{description}
\begin{description}
    \item[\faWrench] \hspace{0.25em} Analog / Mixed Signal Design, MEMS Fabrication \par
\end{description}

\switchcolumn   % goto right side

%-------------------------------------------------------------------
%	EXPERIENCE
%-------------------------------------------------------------------
\cvsection{Experience}

\large{\bold{Control Systems Internship}} \hfill \small\bodyfont{Summers 2021 -- Pres.} \par\smallskip
\vspace{-8pt}
\small\thin{Armored Vehicle Fire Control} \hfill \small\thin{Picatinny Arsenal, NJ}
\setstretch{0.9}

\normalsize\rm Managed the re-engineering of an unmanned ground vehicle electrical system to include failure modes. Investigated the adaptation of existing hardware to an open Ethernet standard. Supported the development of next-generation fire control for medium caliber systems.

\hspace{0pt}
\vspace{-24pt}

\divider

\large{\bold{Graduate Teaching Assistant}} \hfill \small\bodyfont{January 2023 -- Pres.} \par\smallskip
\vspace{-8pt}
\small\thin{Analog Electronics} \hfill \small\thin{RIT}
\setstretch{0.9}

\normalsize\rm Instructed and assisted students in RIT's EE480 Analog Electronics lab. Exercises include SPICE and hardware characterization and simulation of diode circuits, and blutistage BJT/MOSFET amplifiers.

\hspace{0pt}
\vspace{-24pt}

\divider

\large{\bold{Undergraduate Research}} \hfill \small\bodyfont{January 2021 -- Pres.} \par\smallskip
\vspace{-8pt}
\small\thin{Microfluidics MEMS} \hfill \small\thin{RIT \& NSF}
\setstretch{0.9}

\normalsize\rm Developed novel process flow for manufacturing piezoresistive \\diaphragm array with \qty{200}{\nm} thickness. Packaging and testing for use to improve current microfluidic models for pumpless cooling of electronics.

% \hspace{0pt}
% \vspace{-24pt}

% \divider

% \parbox[t]{0.25 \textwidth}{\Large{\bold{RoNetco}} \par\smallskip
% \begingroup
% \setstretch{0.85}
% \normalsize{Information and \\ Electronics Technologies}
% \par\endgroup
% \bodyfont{June 2019 -- September 2020 \\ Ledgewood, NJ}
% }
% \hfill
% \parbox[t]{0.5 \linewidth}{\vspace{-11pt} Designed and documented network room fire suppression system. Drafted AutoCAD drawings for construction. Improved efficiency by 90\% via automating the transition from Windows 7 to 10.}
% \hspace{0pt}
% \medskip

%-------------------------------------------------------------------
%	PROJECTS
%-------------------------------------------------------------------
\cvsection{Projects}

\parbox[t]{0.25 \textwidth}{\large{\bold{RTR Fully-Differential \\Op-Amp}}}
\hfill
\parbox[t]{0.5 \linewidth}{\vspace{-11pt} Verilog Harvard architecture CPU implemented in hardware on Altera FPGA. Wrote Python assembler for full development stack control.}
\hspace{0pt}
\vspace{0pt}

\divider

\parbox[t]{0.25 \textwidth}{\Large{\bold{MSP430 \\ Oscilloscope}}}
\hfill
\parbox[t]{0.5 \linewidth}{\vspace{-11pt} Created a functional portable oscilloscope using Assembly algorithms. Uses capacitive touchpad for input and LEDs and UART for display.}
\hspace{0pt}
\vspace{-12pt}

\divider

\parbox[t]{0.25 \textwidth}{\Large{\bold{Mandelbrot Set \\ Render}}}
\hfill
\parbox[t]{0.5 \linewidth}{\vspace{-11pt} Efficiently rendered Mandelbrot Fractal in C++ with colorization and vector-mapping. Integrating with custom thread pool.}
\hspace{0pt}
\vspace{-12pt}

\end{paracol}

\end{document}
